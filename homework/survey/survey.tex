\documentclass[conference,12pt]{IEEEtran}
\usepackage{pdflscape}
\usepackage{hyperref}
\usepackage{tabularx}
\usepackage{graphicx, subfigure, amsmath} 
\usepackage{pdfpages}
\usepackage[backend=biber,style=ieee]{biblatex}
%\usepackage[section]{placeins}
\addbibresource{References.bib}
\interdisplaylinepenalty=2500

% correct bad hyphenation here
\hyphenation{}


\begin{document}
%
% paper title
\title{Better Than Best Effort}

\author{
\IEEEauthorblockN{Jeremy Wright}
\IEEEauthorblockA{Arizona State University\\jlwrigh1@asu.edu}
}
\maketitle
\begin{abstract}
Vehicle's are quickly becoming distributed comupting platforms on wheels, and
teh integration of these system is essentail for both the safety and comfort of
the passengers. Over the last few decades there has been increasing integrtion
of these systems to perform highly advanced functions, but never losing thier
core safety and realibility guiarntees. This survey will look at the 4 most
popular industial networks in vehciles, how they differ from the enthernet based
protocols we are used to in the internet domain, and how the physical layers and
software layers interact to make a truely safe system. 
\end{abstract}

\begin{IEEEkeywords}
CAN, LIN, MOST, FlexRay, safety, reliability
\end{IEEEkeywords}

\section{Introduction}

Something flashes across the road. Instinctively you slam on the brakes. ABS
kicks in to preserve tire contact with the road. Your body begins to fly
forward, your head towards the steering wheel. Inertial sensors deploy a series
of events collision management events. The fuel pump is shut down to reduce the
risk of fire. Body dynamic sensors adjust suspension forces to keep the vehicle
level, and in a safe driving position. The vehicle signals emergency responders
of a crash event with the current GPS location. You are caught by the air bag.
The crash is over. You are safe. 

This is the environment of industrial networks.
The hard real-time communication channels where failure means people die.
Advanced made in this area have increased vehicle safety, reduced vehicle
weight, and improved overall comfort of the system. Many of us are familiar with
internet application network, such as TCP, UDP and Ethernet.  These routing
protocols were originally developed to reduce the ability of an attacking nation
to take down the United States' communication infrastructure. Ethernet is fault
tolerant and employs a concept called best effort routing, where packets are
routed to a destination in the most likely path for success.  Packets have
a time to live, and if they don�t reach the destination in time, drop off the
network. Consider since a �best effort� approach in a vehicle. If the music
player is shipping data over the vehicle network, and the air bag deploy message
cannot get any bandwidth� catastrophe. Instead of best effort, industrial
protocols take a different approach focused on safety, reliability, and
guaranteed delivery. This focus usually is a tradeoff for performance.  You may
not be able to ship a great deal of data, but what data can be sent is sent
reliability since, when the air bag needs to deploy, being late is useless.

This survey will discuss the industrial physical layers commonly used in
vehicles today: CAN, LIN and MOST, and the software protocols that run over
them, ISO, FlexRay, and MOST. Vehicle network design takes an approach similar
to a VoIP network than how one would design a network primarily for email, and
internet.  In these types of networks latency and jitter are key functions that
translate to real physical requirements. This is typically dealt with in
2 methods Time-Triggered messaged, and Priority Messages. Time Triggered
messages are used when the physical later doesn�t support priority based voting,
where as   in the case of CAN, the physical open-collector function implements
a primitive, yet very effective voting scheme for sending data. This survey will
discuss how messages are segregated into time trigger, or priority networks.
The 4 types of networks running in the vehicle: comfort, chassis, engine, and
describe how these functions work together to provide an immersive infotainment
experience, while keeping our vehicles safer than ever.  

Vehicle functions are
segregated into discrete Electronic Control Units (ECUs) distributed around the
vehicle. These ECUs provide raw sensor data, engine statistics, trouble code
information, and even dynamic suspension information, all in real-time. Before
the advent of CAN in 1986, Industrial protocols allow networked ECUs to
communicate in our vehicles, and airplanes. Protocols such as CAN, LIN, or AFDX,
and their associated physical layers CAN, UART, Ethernet allow for reliable,
scalable, and safe operation of electronic systems within our vehicles.  But
what is it that separates an industrial protocol from a non-industrial one.
Namely, reliability, and safety.  Industrial protocols can be evaluated in how
they meet criteria related to confirmed delivery.  Industrial protocols
typically reduce the bandwidth and speed of the network to add extra
reliability. For instance, CAN uses an open collector signaling scheme which
increases the power required to drive the network than the magnetic isolation
technique of Ethernet, but this simple fact implements an OR operation. This OR
operation is used by senders to verify that each bit they send was received by
the rest of the network.  

CAN was the first industrial protocol introduced by
Bosch in 19XX. Software assisted vehicle functions were connected by point to
point wiring.  Once CAN allowed multiple ECUs to be reliably connected on
a common pair of wires, vehicle manufactures saw improved reliability and
reduced vehicle weight.  *17 pounds*.  Industrial portals also allow ECUs to
work cooperatively merging sensor data from multiple end points to achieve
advanced vehicle functions. Today, the X-by-wire systems are the epitome of this
capable of detecting road obstructions and even stopping the vehicle if
necessary.  

So how is reliability defined? Reliability is an ECU knowing that
a message sent was received by its intended receiver. Real-time. Vehicle busses
must allow message to arrive in a deterministic amount of time. Since message
could contain life critical message such as �deploy air bags� such highest
priority message must get through. This is also related to safety, at the
protocol level, message must get though, at the signal level vehicles are
electrically very noisy. Noise can induce eddy currently or other anomalous bits
in the digital networks. The physical layer must protect against these to
achieve safe operation.  This survey will look at 4 ground based vehicle
industrial protocols, MOST, FlexRay, LIN, and CAN examine the network and
physical layers to how they achieve safety, reliability, data integrity and some
of the security implications of each.  Vehicle functions are segregated into
4 function groups 

\section{CAN} 
Can was the first industrial protocol for vehicles, and CAN is used for nearly
all systems of a vehicle, except for the comfort system which require higher
bandwidth than CAN provides. The novel concept introduced by CAN was the bit
priority voting system built into the FlexRay Flexray did some stuff that some
people care about, do I care, probably not, but whatev.

\section{LIN}
Lin bus is a serial bus designed for ultra-low cost low data rate systems.

\section{MOST}
Media oriented service transport was introduced to provided higher data rates
for the comfort systems of a vehicle. MOST is unique in that it� optical fiber.
Optical fiber provides physical safety by its very nature. Since the signals are
light, they aren't susceptible to magnetic fields as copper wires. This
alleviates a higher function handled by more complex systems on the previous
networks. 

\end{document}
